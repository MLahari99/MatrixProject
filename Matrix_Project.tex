\UseRawInputEncoding
\documentclass{beamer}
\usepackage[utf8]{inputenc}
\usepackage{graphicx}
\graphicspath{ {/home/user/Documents/} }
 
\title{MATRIX PROJECT\\EE1390 Introduction to AI and ML}
\author{Submitted by\\Sowmya(EP17BTECH11018)\\Lahari(EP17BTECH11012)}
\institute{Indian Institute of Technology Hyderabad}
\date{14/02/19}
 
 
 
\begin{document}
 
\frame{\titlepage}
 
\begin{frame}
\frametitle{Table of Contents}
\begin{itemize}
\item Geometric Question
\item Matrix Transformation of the Question
\item Solution in the form of matrix
\item Figure of the solution
\end{itemize}
\end{frame}
 
\begin{frame}
\frametitle{Geometric Question}
If a circle C, whose radius is 3,  touches externally the circle $$x^2+y^2+2x-4y = 4$$ at the point (2,2),then find the length of the intercept cut by this circle C on the x-axis.
\end{frame}

\begin{frame}
\frametitle{Matrix Transformation of the Question}
Equation of the circle which externally touches the circle C at a point (2,2) is  $$X^TX+\begin{pmatrix} 2  \ -4\end{pmatrix}X = 4$$ where 
X is point on the circle $$ X = \begin{pmatrix} x \\ y \end{pmatrix}$$
Find the length of intercept cut by circle C on the x-axis.
\end{frame}

\begin{frame}
\frametitle{Solution in the form of matrix}
Let\\
C1 - Centre of the circle whose equation is unknown\\
r - Radius of the circle whose equation is given\\
C - Centre of the circle whose equation is given\\
\end{frame}

\begin{frame}
\frametitle{Solution in the form of matrix}
Given\\
\begin{equation}
X^TX+\begin{pmatrix} 2  \ -4\end{pmatrix}X = 4
\end{equation}
Since we know that $$(X-C)^T(X-C) = r^2$$ as
$$ ||X-C|| = r$$
\begin{equation}
X^TX-2C^TX = r^2 - C^TC
\end{equation}
\end{frame}

\begin{frame}
\frametitle{Solution in the form of matrix}
Comparing equations (1) and (2) we get
\begin{columns}
\column{0.5\textwidth}
$$ -2C^T = \begin{pmatrix}2\ -4 \end{pmatrix}$$
$$ C^T = \frac{-1\times \begin{pmatrix}2\ -4 \end{pmatrix}}{2} $$
 $$C = \frac{-1\times \begin{pmatrix}2\ -4 \end{pmatrix}^T}{2} $$
\begin{equation}
C^TC = \begin{pmatrix}-1\ 2 \end{pmatrix} \begin{pmatrix}-1\\ 2 \end{pmatrix}  = 5
\end{equation}
\column{0.5\textwidth}
$$r^2-C^TC = 4 $$
$$r^2 = C^TC + 4 $$
\begin{equation}
 r = \sqrt{C^TC + 4} 
\end{equation}
\end{columns}
From equations (3) and (4) we get 
$$ r = \sqrt{ 5 + 4 } = 3 $$
\end{frame}

\begin{frame}
\frametitle{Solution in the form of matrix}
Let n be the direction vector of line joining centre C and point(2,2)$$n = \begin{pmatrix}-1\\ 2 \end{pmatrix} - \begin{pmatrix}2\\ 2 \end{pmatrix} = \begin{pmatrix}-3\\ 0 \end{pmatrix}$$\\
We know that when circles touch each other externally at a point, the point and the centres of the circles lie on the same line which implies $$C1 - C = kn $$ where k is some constant.
\end{frame}

\begin{frame}
\frametitle{Solution in the form of matrix}
As we know that when circles touch each other externally the distance between the centres is same as the sum of their radii.
$$|C1 - C| = r1 + r  (since \ r = 3\ and \ radius \ of\ the\ other\ circle\ (r1)\ = 3) $$
$$( C1 - C )^T( C1- C ) = 6^2 $$
$$ (kn)^T(kn) = 36 $$
$$k^2 \times n^Tn = 36 $$
$$ k^2 = \frac{36}{n^Tn} $$
$$ As\ n^Tn = \begin{pmatrix}-3\ 0 \end{pmatrix} \begin{pmatrix}-3\\ 0 \end{pmatrix} = 9 $$
$$ k^2 = \frac{36}{9} = 4 $$
$$ k = \pm 2  ..... (a)$$
\end{frame}

\begin{frame}
\frametitle{Solution in the form of matrix}
Also \begin{equation}C1 = kn + C \end{equation}
$$(C1-\begin{pmatrix}2\\ 2 \end{pmatrix})^T(C1-\begin{pmatrix}2\\ 2 \end{pmatrix}) = 3^2 $$
From (5) $$ ( C + kn - \begin{pmatrix}2\\ 2 \end{pmatrix})^T( C + kn - \begin{pmatrix}2\\ 2 \end{pmatrix}) = 9 $$
$$( \begin{pmatrix}-1\\ 2 \end{pmatrix} + k \begin{pmatrix}-3\\ 0 \end{pmatrix} - \begin{pmatrix}2\\ 2 \end{pmatrix})^T(\begin{pmatrix}-1\\ 2 \end{pmatrix} + k \begin{pmatrix}-3\\ 0 \end{pmatrix} - \begin{pmatrix}2\\ 2 \end{pmatrix}) = 9 $$
$$ \begin{pmatrix}-3k - 3 \ \ 0 \end{pmatrix} \begin{pmatrix}-3k - 3 \\ 0 \end{pmatrix} = 9 $$
$$ ( -3k - 3 )^ 2 + 0^2 = 9 $$
\end{frame}

\begin{frame}
\frametitle{Solution in the form of matrix}
$$ 9(k+1)^2 = 9 $$
$$ (k+1)^2 = 1 $$
$$ k + 1 = \pm 1 $$
$$ k = 0,-2  ......(b)$$ 
From (a) and (b) $$ k = -2 $$
$$ C = \begin{pmatrix}-1\\ 2 \end{pmatrix} + k\begin{pmatrix}-3\\ 0 \end{pmatrix} = \begin{pmatrix}-1\\ 2 \end{pmatrix} + -2\begin{pmatrix}-3\\ 0 \end{pmatrix} $$ 
$$C = \begin{pmatrix}5\\ 2 \end{pmatrix}$$
\end{frame}

\begin{frame}
\frametitle{Solution in the form of matrix}
We now know the centre of the circle C and its radius(r1 = 3 given). So the equation of the circle is 
$$(Y - C)^T(Y - C) = 3^2 $$
where Y is a point on the circle and as $$|Y - C| = 3$$
$$Y^TY - 2C^TY + C^TC = 9 $$
$$Y^TY - 2\begin{pmatrix}5\ 2 \end{pmatrix}Y + \begin{pmatrix}5\ 2 \end{pmatrix}\begin{pmatrix}5\\ 2 \end{pmatrix} = 9 $$
\begin{equation}
Y^TY - 2\begin{pmatrix}5\ 2 \end{pmatrix}Y + 20 = 0 
\end{equation}
\end{frame}

\begin{frame}
\frametitle{Solution of the form of matrix}
To get the x-intercept we take general point on x-axis, substitute it in the circle equation and solve for points and then find distance between them.
$$ Substituting \ Y = \begin{pmatrix}x\\ 0 \end{pmatrix} \ in \ (6) $$
$$\begin{pmatrix}x\ 0 \end{pmatrix}\begin{pmatrix}x\\ 0 \end{pmatrix} - 2\begin{pmatrix}5\ 2 \end{pmatrix}\begin{pmatrix}x\\ 0 \end{pmatrix} + 20 = 0 $$
$$ x^2 -10x + 20 = 0 $$
$$ x = \frac{10\pm\sqrt{10^2-4\times20}}{2} $$
$$ x = \frac{10\pm\sqrt{20}}{2} $$
$$ x = 5 \pm \sqrt{5} $$
$$ X-intercept = \Delta{x} = 2\sqrt{5} $$
\end{frame}
 
\begin{frame}
\frametitle{Figure of the Solution}
\begin{figure}
\includegraphics[width = 320pt]{fig.jpeg}
\end{figure}
\end{frame}
 
\end{document}