\UseRawInputEncoding
\documentclass{beamer}
\usepackage[utf8]{inputenc}
\usepackage{graphicx}
\graphicspath{ {/home/user/Documents/} }
 
\usetheme{Copenhagen}
\title{MATRIX PROJECT\\EE1390 Introduction to AI and ML}
\author{By\\Sowmya and Lahari}
\institute{Indian Institute of Technology Hyderabad}
\date{14/02/19}
 
 
 
\begin{document}
 
\frame{\titlepage}
 
\begin{frame}
\frametitle{Table of Contents}
\begin{itemize}
\item Geometric Question
\item Matrix Transformation of the Question
\item Solution in the form of matrix
\item Figure of the solution
\end{itemize}
\end{frame}
 
\begin{frame}
\frametitle{Geometric Question}
If a circle \textbf{C}, whose radius is 3,  touches externally the circle $$x^2+y^2+2x-4y = 4$$ at the point (2,2),then find the length of the intercept cut by this circle \textbf{C} on the x-axis.
\end{frame}

\begin{frame}
\frametitle{Figure}
\includegraphics[width = 300pt]{Fig2}
\end{frame}

\begin{frame}
\frametitle{Matrix Transformation of the Question}
Equation of the circle which externally touches the circle \textbf{C} at a point (2,2) is  $$\textbf{X}^T\textbf{X}+\begin{pmatrix} 2  \ -4\end{pmatrix}\textbf{X} = 4$$ where 
X is point on the circle $$ \textbf{X} = \begin{pmatrix} x \\ y \end{pmatrix}$$
Find the length of intercept cut by circle \textbf{C} on the x-axis.
\end{frame}

\begin{frame}
\frametitle{Solution in the form of matrix}
Let\\
\textbf{C1} - Centre of the circle whose equation is unknown\\
r - Radius of the circle whose equation is given\\
\textbf{C} - Centre of the circle whose equation is given\\
\end{frame}

\begin{frame}
\frametitle{Solution in the form of matrix}
Given\\
\begin{equation}
\textbf{X}^T\textbf{X}+\begin{pmatrix} 2  \ -4\end{pmatrix}\textbf{X} = 4
\end{equation}
Since we know that $$(\textbf{X}-\textbf{C})^T(\textbf{X}-\textbf{C}) = r^2$$ as
$$ ||\textbf{X}-\textbf{C}|| = r$$
\begin{equation}
\textbf{X}^T\textbf{X}-2\textbf{C}^T\textbf{X} = r^2 - \textbf{C}^T\textbf{C}
\end{equation}
\end{frame}

\begin{frame}
\frametitle{Solution in the form of matrix}
Comparing equations (1) and (2) we get
\begin{columns}
\column{0.5\textwidth}
$$ -2\textbf{C}^T = \begin{pmatrix}2\ -4 \end{pmatrix}$$
$$ \textbf{C}^T = \frac{-1\times \begin{pmatrix}2\ -4 \end{pmatrix}}{2} $$
 $$\textbf{C} = \frac{-1\times \begin{pmatrix}2\ -4 \end{pmatrix}^T}{2} $$
\begin{equation}
\textbf{C}^T\textbf{C} = \begin{pmatrix}-1\ 2 \end{pmatrix} \begin{pmatrix}-1\\ 2 \end{pmatrix}  = 5
\end{equation}
\column{0.5\textwidth}
$$r^2-\textbf{C}^T\textbf{C} = 4 $$
$$r^2 = \textbf{C}^T\textbf{C} + 4 $$
\begin{equation}
 r = \sqrt{\textbf{C}^T\textbf{C} + 4} 
\end{equation}
\end{columns}
From equations (3) and (4) we get 
$$ r = \sqrt{ 5 + 4 } = 3 $$
\end{frame}

\begin{frame}
\frametitle{Solution in the form of matrix}
Let \textbf{n} be the direction vector of line joining centre C and point(2,2)$$\textbf{n} = \begin{pmatrix}-1\\ 2 \end{pmatrix} - \begin{pmatrix}2\\ 2 \end{pmatrix} = \begin{pmatrix}-3\\ 0 \end{pmatrix}$$\\
The line joining centres of two externally touching circles includes the point of contact.$$\textbf{C1} - \textbf{C} = k\textbf{n} $$ where k is some constant.
\end{frame}

\begin{frame}
\frametitle{Solution in the form of matrix}
\begin{equation}\textbf{C1} = k\textbf{n} + \textbf{C} \end{equation}
$$(\textbf{C1}-\begin{pmatrix}2\\ 2 \end{pmatrix})^T(\textbf{C1}-\begin{pmatrix}2\\ 2 \end{pmatrix}) = 3^2 $$
From (5) $$ ( \textbf{C} + k\textbf{n} - \begin{pmatrix}2\\ 2 \end{pmatrix})^T( \textbf{C} + k\textbf{n} - \begin{pmatrix}2\\ 2 \end{pmatrix}) = 9 $$
$$( \begin{pmatrix}-1\\ 2 \end{pmatrix} + k \begin{pmatrix}-3\\ 0 \end{pmatrix} - \begin{pmatrix}2\\ 2 \end{pmatrix})^T(\begin{pmatrix}-1\\ 2 \end{pmatrix} + k \begin{pmatrix}-3\\ 0 \end{pmatrix} - \begin{pmatrix}2\\ 2 \end{pmatrix}) = 9 $$
$$ \begin{pmatrix}-3k - 3 \ \ 0 \end{pmatrix} \begin{pmatrix}-3k - 3 \\ 0 \end{pmatrix} = 9 $$
$$ ( -3k - 3 )^ 2 + 0^2 = 9 $$
\end{frame}

\begin{frame}
\frametitle{Solution in the form of matrix}
$$ 9(k+1)^2 = 9 $$
$$ (k+1)^2 = 1 $$
$$ k + 1 = \pm 1 $$
$$ k = 0,-2  $$ 
We require a non-zero value of k so $$k = -2$$
$$ \textbf{C} = \begin{pmatrix}-1\\ 2 \end{pmatrix} + k\begin{pmatrix}-3\\ 0 \end{pmatrix} = \begin{pmatrix}-1\\ 2 \end{pmatrix} + -2\begin{pmatrix}-3\\ 0 \end{pmatrix} $$ 
$$\textbf{C} = \begin{pmatrix}5\\ 2 \end{pmatrix}$$
\end{frame}

\begin{frame}
\frametitle{Solution in the form of matrix}
We now know the centre of the circle \textbf{C} and its radius(r1 = 3 given). So the equation of the circle is 
$$(\textbf{Y} - \textbf{C})^T(\textbf{Y} - \textbf{C}) = 3^2 $$
where \textbf{Y} is a point on the circle and as $$||\textbf{Y} - \textbf{C}|| = 3$$
$$\textbf{Y}^T\textbf{Y} - 2\textbf{C}^T\textbf{Y} + \textbf{C}^T\textbf{C} = 9 $$
$$\textbf{Y}^T\textbf{Y} - 2\begin{pmatrix}5\ 2 \end{pmatrix}\textbf{Y} + \begin{pmatrix}5\ 2 \end{pmatrix}\begin{pmatrix}5\\ 2 \end{pmatrix} = 9 $$
\begin{equation}
\textbf{Y}^T\textbf{Y} - 2\begin{pmatrix}5\ 2 \end{pmatrix}\textbf{Y} + 20 = 0 
\end{equation}
\end{frame}

\begin{frame}
\frametitle{Solution of the form of matrix}
To get the x-intercept we take general point on x-axis, substitute it in the circle equation and solve for points and then find distance between them.
$$ Substituting \ \textbf{Y} = \begin{pmatrix}x\\ 0 \end{pmatrix} \ in \ (6) $$
$$\begin{pmatrix}x\ 0 \end{pmatrix}\begin{pmatrix}x\\ 0 \end{pmatrix} - 2\begin{pmatrix}5\ 2 \end{pmatrix}\begin{pmatrix}x\\ 0 \end{pmatrix} + 20 = 0 $$
$$ x^2 -10x + 20 = 0 $$
$$ x = \frac{10\pm\sqrt{10^2-4\times20}}{2} $$
$$ x = \frac{10\pm\sqrt{20}}{2} $$
$$ x = 5 \pm \sqrt{5} $$
$$ X-intercept = \Delta{x} = 2\sqrt{5} $$
\end{frame}
 
\begin{frame}
\frametitle{Figure of the Solution}
\begin{figure}
\includegraphics[width = 300pt]{fig.jpeg}
\end{figure}
\end{frame}
 
\end{document}